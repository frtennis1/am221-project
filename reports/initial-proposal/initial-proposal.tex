%        File: initial-proposal.tex
%     Created: Tue Feb 20 10:00 PM 2018 E
% Last Change: Tue Feb 20 10:00 PM 2018 E
%
\documentclass[letter]{amsart}
\usepackage{../am221}
\usepackage[margin=1in]{geometry}
\linespread{1.1}


\begin{document}



\title{APMTH 221 Final Project Proposal}
\author{Jiafeng (Kevin) Chen \and Francisco Rivera}
\date{\today}

\maketitle

\section{Collaboration}

Francisco and Kevin will work together to research the relevant optimization
techniques helpful to calculating the Wasserstein distance quickly. Then, they
will split up techniques, and through implementations of them, empirically (and
when possible, theoretically) assess their performance. Time permitting, they
will then attempt to create a new algorithm that performs competitively with
state of the art benchmarks.

\section{Problem Statement}

If we are given two probability measures $\mu$ and $\nu$ on metric space
$(M,d)$ with finite moments,
\[ \int_M d(x,x_0)^p d\mu(x) < \infty \text{ and }\int_M d(x,x_0)^p d\nu(x) <
\infty\]
then we define the $p^\text{th}$ Wasserstein distance as
\[ W_p(\mu, \nu) := \left( \inf_{\gamma \in \Gamma(\mu, \nu)} \int_{M \times
M} d(x,y)^p \text{d} \gamma(x,y) \right)^{1/p}\]
where $\Gamma(\mu, \nu)$ is the collection of all measures on $M \times M$ with
marginals $\mu$ and $\nu$.


This can also be thought of as the optimal transport problem. If we think of a
probability distribution as a collection of ``mass'' at different points in $M$
(with density of the mass proportional to the probability measure), then the
Wasserstein distance captures the cost of transforming one distribution into the
other in units of mass times distance.  This analogy is particularly
interpretable if we have a discrete metric space, where probability measures can
be viewed as point masses.

Solving the optimization problem underpinning this metric is in general hard and
remains an unsolved problem. We will aim to test the effectiveness of different
optimization techniques to this problem, with particular focus on discrete
metric spaces.

\section{Motivation}

Numerous important applications exist for the Wasserstein metric. 
% TODO: what are these important applications?

\section{Data}

Because wish to test the effectiveness of Wasserstein distance calculations for
arbitrary distributions, we will generate test distributions using simulation
and do not need to concern ourselves with finding real data. When necessary, we
will use standard distributions found in the literature for consistency.

\section{Deliverables}

We will implement optimization techniques to calculate Wasserstein distances and
attempt a novel algorithm that performs competitively.

\section{Next Steps}

\begin{itemize}

\item We will use linear programs to calculate Wasserstein distances for
discrete metric spaces. We will explore average complexity for randomly
generated test cases.

\item 

\end{itemize}

% TODO: what are we doing next?

\end{document}


