\documentclass[11pt,reqno]{amsart}
\usepackage{../am221}
\linespread{1.2}
\usepackage[margin=1.5in]{geometry}
\title{Optimal Transport Problems}
\newcommand{\R}{\mathbb{R}}
\usepackage{bbm}
\newcommand{\one}{\mathbbm{1}}
\newcommand{\E}{\mathbb{E}}
%\usepackage[margin=1in]{geometry}
\author{Jiafeng Chen\and Francisco Rivera}
\date{\today}
\usepackage{hyperref}
\usepackage[capitalise,nameinlink]{cleveref}

\hypersetup{colorlinks=true, citecolor=Blue, linkcolor=Brown}
\renewcommand{\b}{\mathbf}

\theoremstyle{definition}
\newtheorem{prop}{Proposition}
\theoremstyle{remark}
\newtheorem{rmk}{Remark}
\begin{document}
    \maketitle
    \section{Theoretical foundations}
    The following is a concise review of selected topics in Chapters 2 and 3 in 
    \cite{peyre2017computational}.
    
    Let $\b a \in \R^m_{\ge 0}$ and $\b b \in \R^n_{\ge 0}$ be two vectors of
    nonnegative real numbers such that $\b a^T \one = \b b^T \one = 1$, where
    $a_i$ and $b_j$ represents the amount of ``mass'' at different positions. 
    We are
    interested in a matrix $\b P \in \R^{m \times n}_{\ge 0}$ where $P_{ij}$
    prescribes
    the amount of mass flowing from $a_i$ to $b_j$. Naturally, we must have
    $\sum_j P_{ij} = a_i$ and $\sum_i P_{ij} = b_j$, so that mass is preserved.
    Let $\b C
    \in \R^{m\times n}$ be a \emph{cost matrix}, where $C_{ij}$ represents the
    cost of transporting one unit of $a_i$ to $b_j$. The \emph{Kantorovich
    optimal transport problem} is the following linear program: \begin{align*}
    \min_{\b P \in  \R^{m \times n}_{\ge 0}} \, & \sum_{i,j} P_{ij} C_{ij} \\
    \text{subjected to } & \b P \one = \b a \\
    & \b P^T \one = \b b
    \end{align*}
    
    \begin{rmk}[A probabilistic view]
    Let $X,Y$ be discrete random variables with support $\b
    x \in \mathcal X^m$ and $\b y \in \mathcal Y^n$, such that their marginal
    distributions are
    prescribed by $P(X = x_i) = a_i$ and $P(Y =
    y_j) = b_j$. Let $P_{ij} = P(X = x_i, Y = y_j)$, making $\b P = (P_{ij})_
    {ij}$ a joint
    distribution. Let $C: \mathcal X \times \mathcal Y \to \R$ be a cost
    function. Then the
    Kantorovich problem is equivalent to the following:\[
    \min_{\b P}\, \E_{(X,Y) \sim \b P}\bk{C(X,Y)},
    \]
    where the expectation is taken over valid joint distributions of $(X,Y)$
    respecting the marginal distributions. Note that when $\mathcal X = \mathcal
    Y = \R^d$ and $C(x,y) = \norm{x-y}^p$, then the optimal value of the above
    problem for $X,Y$ is the $p$-Wasserstein distance between the marginal
    distributions $\mu_X,\mu_Y$ (up to a power of $1/p$), for discrete,
    $d$-dimensional distributions with finite support.
    \end{rmk}
    
    It can be shown that the dual problem of the above LP takes the following
    form: \begin{align*}
    \max_{(\b f, \b g) \in \R^{m} \times \R^n} \, & \sum_i a_i f_i + \sum_j b_j
    g_j
    \\ 
    \text{subjected to } & f_i + g_j \le C_{ij}\, \text{ for all $i,j$}.
    \end{align*}
    The dual problem has a neat economic intuition. Suppose Kevin needs to
    transport $\b a$ to $\b b$ but does not understand how. Francisco, his
    profit-maximizing colleague, offers him a deal, where Kevin will pay $f_i$
    for Francisco to
    pick up a unit of mass at $a_i$ and pay $g_j$ for Francisco to dropoff a
    unit of mass at $b_j.$ At minimum, Kevin knows that if any $f_i + g_j > C_
    {ij}$, then Francisco is ripping him off, as transporting one unit from
    $a_i$ to $b_j$ costs exactly $C_{ij}$. If, on the other hand, $f_i + g_j
    \le C_{ij}$ is satisfied, then given any $\b P$, we have \[
    \sum_{i,j} C_{ij}P_{ij} \ge \sum_{i,j} f_i P_{ij} + \sum_{i,j} g_j P_{ij} =
    \sum_i f_i a_i + \sum_j g_j b_j \tag{Weak duality},
    \]
    which means that Kevin cannot lose by taking Francisco's deal. Strong
    duality, which indeed holds, here imply that Francisco's optimal profit is
    zero, and the trade is fair.
    
    % Note that, fixing $\b f$, the optimal choice of $g_j$ is \[
    % g_j = \min_i (C_{ij} - f_i).
    % \]
    % Let $\b g = \b f^{\b C}$ where $(\b f^{\b C})_j = \min_i (C_{ij} - f_i)$,
    % which is called a $\b C$-transform of $\b f$; we may reformulate the dual
    % problem in terms of only one choice variable $\b f$.
    
    Given optimal solutions to the primal and dual, $\b P$, $(\b f, \b g)$,
    respectively, \emph{complementary slackness} implies that \[
    P_{ij} (C_{ij} - f_i - g_j) = 0 \tag{Complementary slackness}
    \]
    holds, which means that $\b P_{ij} \neq 0 \implies C_{ij} = f_i - g_j$ and
    vice versa. 
        
    \subsection{Dual-ascent algorithms and Hungarian algorithm}
    
    Let $S \subset \{1,\ldots,m\}$ and $S' \subset \{1,\ldots,n\}.$ Let $\one_S$
    be a vector of zeros with ones at location prescribed by $S$. Given $\b f,
    \b g$ a feasible pair for the dual, call $(i,j)$ a balanced pair if the
    condition $f_i + g_j = C_{ij}$
    binds, and inactive otherwise. Then 
    \begin{prop}
    
        Let $(\tilde {\b f}, \tilde {\b g}) = (\b f, \b g) + \epsilon (\one_S,
        -\one_{S'})$. Then $(\tilde {\b f}, \tilde {\b g})$ is dual feasible for
        a sufficiently small $\epsilon$ if for all $i$, \[
        (i,j) \text{ is balanced } \implies \text{$j \in S'$}.
        \]
    \label{prop:feasible}
    \end{prop}
    \begin{proof}
        For all $i\in S$, consider \[
        \epsilon_i = \min_{j: (i,j) \text{ inactive}} C_{ij} - f_i - g_j
        \]
        Let $\epsilon = \min_i \epsilon_i$. If $(i,j)$ is inactive, then \[
        \tilde f_i + \tilde g_j \le f_i + \epsilon + g_j\le C_{ij},
        \]
        which continues to be inactive. If $(i,j)$ is balanced, then $\tilde
        f_i + \tilde g_j = f_i + g_j = C_{ij}$. In either case, $\tilde {\b f},
        \tilde {\b g}$ continues to be feasible.
    \end{proof}
    
    \begin{prop}
    \label{prop:dualascent}
        If a dual feasible solution $(\b f, \b g)$ is not optimal, then there
        exists $S,S'$ such that $(\tilde {\b f}, \tilde {\b g}) = (\b f, \b g) + \epsilon (\one_S,
        -\one_{S'})$ is feasible for small enough $\epsilon$ with a strictly
        better objective.
    \end{prop}
    
    \begin{proof}
        Let $\mathcal B$ be the set of balanced edges $(i,j)$, and consider the
        bipartite graph on $\{1,\ldots,m\} \sqcup \{1,\ldots,n\}$ with edges in
        $\mathcal B$. Let there be a source node $s$ with capacitated edges to
        $\{1,\ldots,m\}$ with capacity $a_i$, and a sink node $t$ analogously
        with capacities $b_j$. Consider the maximum flow $\b F$ on the network
        via Ford-Fulkerson. If the throughput of the flow is one, then let $P_
        {ij} = F_{ij}$, which generates a primal feasible solution $\b P$.
        Strong duality would imply that $\b P, (\b f, \b g)$ are both optimal.
        
        Thus we may assume that the throughput is less than one. We label the
        nodes reached from $s$ where $\b F$ does not saturate capacity, as well
        as the nodes that contribute flow to any labeled nodes. Store the
        labeled nodes in $Q$. Since the graph is bipartite, $Q$ can be split
        into $S,S'$. Note that if $(i,j)$ is balanced and $i\in S$, then $j\in
        S'$. By \cref{prop:feasible}, $(\tilde {\b f}, \tilde {\b g})$ so
        defined is feasible. We wish to show that \[
        \one_S^T \b a - \one_{S'}^T \b b > 0.
        \]
        We readily conclude that this is indeed the case by accounting for the
        flow on the network.
    \end{proof}
    
    \cref{prop:dualascent} provides a template for the celebrated Hungarian
    algorithm \cite{kuhn2010hungarian}. The Hungarian algorithm solves the
    Kantorovich problem in the case when both $\b a, \b b \propto \one$. In a
    probability context for the Wasserstein distance between empirical measures,
    if the measures $\mu, \nu$ are both $d$-dimensional with $n$ observations,
    the Hungarian algorithm is $O(n^3)$ for $d>1$ and $O(n\log n)$ for\footnote{In the $d=1$ case, the optimal transport problem reduces to
    sorting.} $d =
    1$
    \cite{bernton2017inference}.
    
    
    \section{Application value}
     
    \cite{bernton2017inference} is an example of recent work applying the
    Wasserstein distance to statistics and machine learning. 
    % Restricting to a subclass of problems called \emph{matching problems}, where
    % $\b a, \b b \propto \one$.

\bibliographystyle{alpha}
\bibliography{../optimal-transport}
\end{document}
