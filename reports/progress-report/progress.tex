\documentclass[11pt]{amsart}
\usepackage{../am221}
\linespread{1.2}
\title{Optimal Transport Problems}
\newcommand{\R}{\mathbb{R}}
\usepackage{bbm}
\newcommand{\one}{\mathbbm{1}}
\newcommand{\E}{\mathbb{E}}
\usepackage{cleveref}
\author{Jiafeng Chen\and Francisco Rivera}
\date{\today}
\usepackage{hyperref}
\hypersetup{colorlinks=true, citecolor=Blue, linkcolor=blue}
\renewcommand{\b}{\mathbf}

\theoremstyle{remark}
\newtheorem{rmk}{Remark}
\begin{document}
    \maketitle
    \section{Theoretical foundations}
    The following is a concise review of Chapters 2 and 3 in 
    \cite{peyre2017computational}.
    
    Let $\b a \in \R^m_{\ge 0}$ and $\b b \in \R^n_{\ge 0}$ be two vectors of
    nonnegative real numbers such that $\b a^T \one = \b b^T \one = 1$, where
    $a_i$ and $b_j$ represents the amount of ``mass'' at different positions. 
    We are
    interested in a matrix $\b P \in \R^{m \times n}_{\ge 0}$ where $P_{ij}$
    prescribes
    the amount of mass flowing from $a_i$ to $b_j$. Naturally, we must have
    $\sum_j P_{ij} = a_i$ and $\sum_i P_{ij} = b_j$, so that mass is preserved.
    Let $\b C
    \in \R^{m\times n}$ be a \emph{cost matrix}, where $C_{ij}$ represents the
    cost of transporting one unit of $a_i$ to $b_j$. The \emph{Kantorovich
    optimal transport problem} is the following linear program: \begin{align*}
    \min_{\b P \in  \R^{m \times n}_{\ge 0}} \, & \sum_{i,j} P_{ij} C_{ij} \\
    \text{subjected to } & \b P \one = \b a \\
    & \b P^T \one = \b b
    \end{align*}
    
    \begin{rmk}[A probabilistic view]
    Let $X,Y$ be discrete random variables with support $\b
    x \in \mathcal X^m$ and $\b y \in \mathcal Y^n$, such that their marginal
    distributions are
    prescribed by $P(X = x_i) = a_i$ and $P(Y =
    y_j) = b_j$. Let $P_{ij} = P(X = x_i, Y = y_j)$, making $\b P = (P_{ij})_
    {ij}$ a joint
    distribution. Let $C: \mathcal X \times \mathcal Y \to \R$ be a cost
    function. Then the
    Kantorovich problem is equivalent to the following:\[
    \min_{\b P}\, \E_{(X,Y) \sim \b P}\bk{C(X,Y)},
    \]
    where the expectation is taken over valid joint distributions of $(X,Y)$
    respecting the marginal distributions. Note that when $\mathcal X = \mathcal
    Y = \R^d$ and $C(x,y) = \norm{x-y}^p$, then the optimal value of the above
    problem for $X,Y$ is the $p$-Wasserstein distance between the marginal
    distributions $\mu_X,\mu_Y$ (up to a power of $1/p$), for discrete,
    $d$-dimensional distributions with finite support.
    \end{rmk}
    
    It can be shown that the dual problem of the above LP takes the following
    form: \begin{align*}
    \max_{(\b f, \b g) \in \R^{m} \times \R^n} \, & \sum_i a_i f_i + \sum_j b_j
    g_j
    \\ 
    \text{subjected to } & f_i + g_j \le C_{ij}\, \text{ for all $i,j$}.
    \end{align*}
    The dual problem has a neat economic intuition. Suppose Kevin needs to
    transport $\b a$ to $\b b$ but does not understand how. Francisco, his
    profit-maximizing colleague, offers him a deal, where Kevin will pay $f_i$
    for Francisco to
    pick up a unit of mass at $a_i$ and pay $g_j$ for Francisco to dropoff a
    unit of mass at $b_j.$ At minimum, Kevin knows that if any $f_i + g_j > C_
    {ij}$, then Francisco is ripping him off, as transporting one unit from
    $a_i$ to $b_j$ costs exactly $C_{ij}$. If, on the other hand, $f_i + g_j
    \le C_{ij}$ is satisfied, then given any $\b P$, we have \[
    \sum_{i,j} C_{ij}P_{ij} \ge \sum_{i,j} f_i P_{ij} + \sum_{i,j} g_j P_{ij} =
    \sum_i f_i a_i + \sum_j g_j b_j \tag{Weak duality},
    \]
    which means that Kevin cannot lose by taking Francisco's deal. Strong
    duality, which indeed holds, here imply that Francisco's optimal profit is
    zero, and the trade is fair.
    
    % Note that, fixing $\b f$, the optimal choice of $g_j$ is \[
    % g_j = \min_i (C_{ij} - f_i).
    % \]
    % Let $\b g = \b f^{\b C}$ where $(\b f^{\b C})_j = \min_i (C_{ij} - f_i)$,
    % which is called a $\b C$-transform of $\b f$; we may reformulate the dual
    % problem in terms of only one choice variable $\b f$.
    
    Given optimal solutions to the primal and dual, $\b P$, $(\b f, \b g)$,
    respectively, \emph{complementary slackness} implies that \[
    P_{ij} (C_{ij} - f_i - g_j) = 0
    \]
    holds, which means that $\b P_{ij} \neq 0 \implies C_{ij} = f_i - g_j$ and
    vice versa. 

\bibliographystyle{alpha}
\bibliography{../optimal-transport}
\end{document}
